\documentclass{article}
\usepackage[utf8]{inputenc}
\usepackage[margin=1.1in]{geometry}
\usepackage{graphicx}
\usepackage{tikz}
\usepackage{caption}
\usepackage{textgreek}
\usepackage{minted}
\usepackage{enumitem}
\usepackage{hyperref}
\usepackage{ulem}
\begin{document}
\title{Teórica 02}
\maketitle
\section{Sobre o que se trabalha?}

\begin{itemize}
\item A representação intermédia geralmente é uma estrutura de dados (p.e. Dicionário);
\item O texto é dado por uma lista de \textit{chars};
\item O que pretendemos obter a partir do texto é uma lista de \textit{tokens};
\item Um \textit{token} possui semântica associada.
\end{itemize}
\section{Expressões Regulares (RegEx)}
Podemos utilizar o site \href{https://regex101.com}{regex101} para treinarmos as expressões e perceber o seu conteúdo.\\
 \subsection{Construtores}

\begin{itemize}
\item \textbf{a} - caractere \textit{a};
\item \textbf{ab} - caractere \textit{a} seguido pelo caractere \textit{b};
\item \textbf{()} - agrupamento de caracteres;
\end{itemize}
\\
 Vamos então representar um ano em regex:\\
 \begin{minted}{python}
('0'|...|'9')('0'|...|'9')('0'|...|'9')('0'|...|'9')
\end{minted}
 
\begin{itemize}
\item \textbf{a?} - caractere \textit{a} é opcional, ou seja, pode ou não surgir;
\item \textbf{a*} - caractere \textit{a} surge 0 ou mais vezes repetidamente;
\item \textbf{a+} - caractere \textit{a} surge 1 ou mais vezes repetidamente;
\end{itemize}
\\
 Vamos então representar um inteiro, independentemente da sua dimensão, em regex:\\
 \begin{minted}{python}
Int = ('-'|'+')?('0'|...'9')+
\end{minted}
 
\begin{itemize}
\item \texttt{\textbackslash} - retira o significado especial de um caractere passando a ver-se esse elemento como um caractere normal.
\end{itemize}
 Como podemos representar uma string binária?\\
 \begin{minted}{python}
(0|1)*
\end{minted}
 E uma string binária que não contém a string '011' em nenhuma parte da string?\\
 \begin{minted}{python}
1*(0|0+1)*
\end{minted}
 
\begin{itemize}
\item \textbf{[ ]} - classe de caracteres, pode, por exemplo representar um intervalo contendo, no seu interior, 'x-y' (a ordem é dada pela tabela ASCII)
\end{itemize}
\\
 Assim sendo, podemos aprimorar o regex que representa um inteiro:\\
 \begin{minted}{python}
('-'|'+')?[0-9]*
\end{minted}
 
\begin{itemize}
\item \{min, max\} - no mínimo, repete 'min' vezes, no máximo, repete 'max' vezes. Se tiver apenas um elemento, repete esse número de vezes. Se não tiver valor à esquerda é 0 e, se não tiver à direita tende para infinito.
\end{itemize}
\\
 Vamos representar uma data e um número real em regex:\\
 \begin{minted}{python}
data = [0-9]{4}-(0[1-9]|1[0-2])-[0-9]{2}
real = (\+|\-)?[0-9]*(\.[0-9]+)?
\end{minted}
\end{document}